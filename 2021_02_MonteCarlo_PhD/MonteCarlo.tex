\documentclass[aspectratio=43]{beamer}
\usepackage[latin1]{inputenc}
\usepackage{amsmath}
\usepackage{amsfonts}
\usepackage{amssymb}
\usepackage{makeidx}
\usepackage{graphicx}
\usepackage{array}

% Customization
\mode<presentation>{
\usetheme{CambridgeUS}
\usecolortheme{dolphin}
\setbeamertemplate{navigation symbols}{}
}

% Define colors
\definecolor{darkgreen}{rgb}{0.0, 0.5, 0.13}
\definecolor{darkblue}{rgb}{0.0, 0.0, 0.55}
\definecolor{darkred}{rgb}{0.55, 0.0, 0.0}

%************************************************************************************************************

% Title and author
\title[QCD and Monte Carlo]{QCD and Monte Carlo}
\author{\textbf {Jes\'us Urtasun Elizari}}
\date{Milan, February 2021}

\begin{document}

% Front slide
\begin{frame}
	
	%\maketitle
	\vspace{1.0 cm}
	
	\center{\color{blue}QCD and Monte Carlo event generators}
	
	\vspace{0.25 cm}
	\center{Monte Carlo course seminar - Milan, February 2021}

	\begin{figure}
		\minipage{1\textwidth}
		\includegraphics[width = 3.0 cm]{plots/logo_unimi.png}
		\hfill
		\includegraphics[width = 3.0 cm]{plots/logo_infn.png}
		\hfill
		\includegraphics[width = 3.0 cm]{plots/logo_erc.png}
		\endminipage
	\end{figure}

	\vspace{1.0 cm}

\end{frame}

% Introduction
\begin{frame}

	\frametitle{Outline}
	
	\begin{enumerate}
		\item {\color{blue}Hadron collisions and strong interactions}
		\begin{itemize}
			\item Hadron collisions and strong interactions
			\item Renormalization group
			\item QCD factorization
		\end{itemize}
		\item {\color{blue}MC and Parton Showers}
		\begin{itemize}
			\item Collinear factorization
			\item Final state radiation
			\item Initial state radiation
		\end{itemize}
		\item {\color{blue}Hadronization: some basics}
		\begin{itemize}
			\item \footnotesize Large number of colors approximation
			\item \footnotesize Hadronization models
		\end{itemize}
	\end{enumerate}
	
\end{frame}

% Strong interactions I
\begin{frame}

	\frametitle{Strong interactions}
	\framesubtitle{QCD from $e^{+}e^{-}$ annihilation}

	Quantum Chromodynamics (QCD) $\rightarrow$ theory describing the interaction between quarks and gluons (strong interactions)
	\begin{figure}
		\includegraphics[width = 5 cm]{plots/ee_hadrons.png}
	\end{figure}
 
	QCD arises already from $e^{+}e^{-}$ annihilation $\rightarrow R_{0}$ ratio

	\begin{columns}
	
		\column{0.5\textwidth}
	
		\begin{figure}
			\includegraphics[width = 6 cm]{plots/eq_R0.png}
		\end{figure}
	
		\column{0.5\textwidth}
	
		\begin{enumerate}
			\item \footnotesize Color factor (3 color for each quark)
			\item \footnotesize Sum over charges of different flavors
			\item \footnotesize Threshold and higher order corrections
		\end{enumerate}	

	\end{columns}

\end{frame}

% Renormalization group I
\begin{frame}

	\frametitle{Strong interactions}
	\framesubtitle{Renormalization group}
	
	\begin{itemize}
		\item Running coupling given by Renormalization Group Equation (RGE)
		\begin{equation}
			{\color{blue}\mu\frac{d\alpha_{s}(\mu)}{d\mu} = \beta(\alpha_{s}(\mu)) = -\sum_{n = 0}^{\infty} \beta_{n} \Big( \frac{\alpha_{s}}{\pi} \Big)^{n + 1}} \nonumber
		\end{equation}
		\item Coupling {\color{blue}$\alpha_{s}$} evolves with scale {\color{blue}$\mu$} as given by RGE $\rightarrow$ LO behavior driven by $\beta_{0}$
		\item $\beta_{0}^{\textrm{QCD}} > 0 \implies$ weakly coupled at large energies, asymptotic freedom
		\item $\beta_{0}^{\textrm{QED}} < 0 \implies$ strongly coupled at large energies, UV divergent!	
	\end{itemize}

\end{frame}

% Renormalization group II
\begin{frame}

	\frametitle{Strong interactions}
	\framesubtitle{Renormalization group}

	\begin{columns}	
	
	\column{0.5\textwidth}
	
	\begin{figure}
		\includegraphics[width = 5 cm]{plots/qcd_coupling.png}
	\end{figure}
	
	\column{0.5\textwidth}
	
	\begin{itemize}
		\item \footnotesize Running coupling given by Renormalization Group Equation (RGE)
		\begin{equation}
			\alpha_{s}(\mu) = \frac{1}{\beta_{0} \log\big( \frac{\mu^{2}}{\Lambda_{s}^{2}}\big)} \nonumber
		\end{equation}
		\item \footnotesize $\beta_{0}$ LO of the $\beta$ function, is $ > 0$
		\item \footnotesize $\Lambda_{s}$, parameter that defines value of the coupling at large scales
	\end{itemize}
	
\end{columns}
	
	\vspace{1cm}
	\center QCD is weakly coupled for $\mu >> \Lambda_{s} \longrightarrow$ asymptotically free
	\center \color{red} Perturbative Quantum Chromodynamics (pQCD)

\end{frame}

% Factorization theoren
\begin{frame}

	\frametitle{Factorization theorem}
	\framesubtitle{QCD factorization}
	
	\center \footnotesize LHC processes $H_{1} + H_{2} \rightarrow \textrm{F}$	
	\begin{figure}
		\includegraphics[width = 7 cm]{plots/factorization_1.png}
	\end{figure}
	
	\footnotesize {Separate process {\color{blue}PDFs} and {\color{red} partonic (hard) interaction}	
	\begin{equation}
		\sigma^{\textrm{F}}(p_{1}, p_{2}) = \sum_{\alpha, \beta}
		\int_{0}^{1} dx_{1} dx_{2} \; {\color{blue} f_{\alpha}(x_{1}, \mu_{F}^{2}) \ast f_{\beta}(x_{2}, \mu_{F}^{2})}
		\; \ast \;  
		{\color{red}\hat{\sigma}^{\textrm{F}}_{\alpha \beta}(x_{1}p_{1}, x_{2}p_{2}, \alpha_{s}(\mu_{R}^{2}), \mu_{F}^{2})} \nonumber
	\end{equation}}
		
\end{frame}

% Parton showers I
\begin{frame}

	\frametitle{Parton showers}
	\framesubtitle{MC Parton showers}
	
	\footnotesize Partons in the initial and final state emit radiation. Initial state Radiation (ISR) and Final State Radiation (FSR) model by Monte Carlo (MC) shower algorithms
	
	\vspace{0.1 cm}
	\center \color{red} Shower Monte Carlo programs (HERWIG, PYTHIA)
	\vspace{0.15 cm}
	
	\begin{itemize} 
		\item Libraries for computing SM and BSM cross sections
		\item Shower algorithms produce a number of enhanced
		coloured parton emissions \\
		to be added to the hard process
		\item Hadronization models, underlying event, decays of unstable hadrons, etc
	\end{itemize}

\end{frame}

% Parton showers II
\begin{frame}

	\frametitle{Parton showers}
	\framesubtitle{Collinear limit}
	
	\begin{itemize} 
		\item \footnotesize QCD emission processes are enhanced in the collinear limit ($\theta$ small)
		\item \footnotesize $\sigma$ dominated by collinear splittings $q \rightarrow qg, g \rightarrow gg, g \rightarrow q\bar{q}$
	\end{itemize}
	
	\begin{figure}
		\includegraphics[width = 6 cm]{plots/collinear_factorization.png}
	\end{figure}
	
	\footnotesize Collinear factorization $\longrightarrow$ The cross section factorizes
	into the product of a tree-level cross section and a splitting
	factor out tree level amplitude and splitting
	\begin{figure}
		\includegraphics[width = 7 cm]{plots/eq_factorization_theorem.png}
	\end{figure}

	\begin{figure}
		\includegraphics[width = 5 cm]{plots/eq_factorization_ps.png}
	\end{figure}

\end{frame}

% Parton showers III
\begin{frame}

	\frametitle{Parton showers}
	\framesubtitle{Kinematics of splitting}
	
	\begin{figure}
		\includegraphics[width = 5 cm]{plots/shower_kinematics.png}
	\end{figure}
	
	\footnotesize Kinematics of splitting given by $(t, z, \phi)$
	\begin{itemize}
		\item $t$: parameter with dimensions of energy that vanish in the collinear limit \\
		\begin{itemize}
			\item Virtuality $t = (k + l)^{2} \approx z(1 - z)E^{2}\theta^{2}$
			\item Transverse momentum $t = k_{\perp}^{2} = l_{\perp}^{2} \approx z^{2}(1 - z)^{2}E^{2}\theta^{2}$
			\item Hardness $t = E^{2}\theta^{2}$
		\end{itemize}
		\item $z$: fraction of energy of radiated parton $z = \frac{k^{0}}{k^{0} + l^{0}}$
		\item $\phi$ represents azimuth of the $k, l$ plane
	\end{itemize}
	
\end{frame}

% Parton showers IV
\begin{frame}

	\frametitle{Parton showers}
	\framesubtitle{AP splitting functions}
	
	\footnotesize Factorization holds for small angles $\rightarrow$ small $t$ variable \\
	\footnotesize Difference in the splitting $\rightarrow$ Altarelli-Parisi splitting functions (singular in $z \rightarrow 0, 1$)
	
		
	\begin{columns}
	
		\column{0.4\textwidth}
			 
		\begin{figure}
			\includegraphics[width = 5 cm]{plots/AP_splitting.png}
		\end{figure}
		
		\column{0.6\textwidth}
		
		\begin{figure}
			\includegraphics[width = 6.5 cm]{plots/AP_iteration_plot.png}
		\end{figure}

	\end{columns}

	\vspace{0.4cm}
	
	\footnotesize We can proceed in an iterative way	
	\begin{figure}
		\includegraphics[width = 9 cm]{plots/AP_iteration_eq.png}
	\end{figure}

	\footnotesize Angles become small, maintaining a strong ordering relation $\theta' >> \theta \rightarrow 0$
	
\end{frame}

% Parton showers V
\begin{frame}
	
	\frametitle{Parton showers}
	\framesubtitle{Exclusive final state}
	
	\footnotesize To describe exclusive final state $\rightarrow$ sum perturbative expansion to all orders in $\alpha_s$ \\
	
	\begin{figure}
		\includegraphics[width = 9 cm]{plots/exclusive_ll.png}
	\end{figure}
	
	\center \footnotesize Limit to the most singular terms, in ordered sequence of angles \\
	\center \color{red} Collinear approximation $\longrightarrow$ Leading log approximation

\end{frame}

% FSR I
\begin{frame}

	\frametitle{Final state radiation MC}
	\framesubtitle{General structure}
	
	\footnotesize Approximated description of a hadronic final state \\
	\footnotesize Model a given hard scattering with arbitrary number of enhanced radiations

	\begin{itemize} 
		\item Choose hard interaction with specified Born kinematics
		\item Consider all possible tree-level splittings for each coloured parton
		\item Assign the variables ($t, z, \phi$) at each splitting vertex, $t$ ordered in decreasing way
		\item At each splitting vertex assign the weight 
		\begin{figure}
			\includegraphics[width = 2.5 cm]{plots/AP_weight.png}
		\end{figure}
		\item Each line has a weight known as Sudakov factor
		\begin{figure}
			\includegraphics[width = 7 cm]{plots/sudakov.png}
		\end{figure}
	\end{itemize}

\end{frame}

% FSR II - Formal representation of a shower
\begin{frame}

	\frametitle{Final state radiation MC}
	\framesubtitle{Formal representation of a shower}
	
	\footnotesize Graphical notation for the representation of a shower \\
	
	\begin{columns} 
		
		\column{0.5\textwidth}
		
		\begin{figure}
			\includegraphics[width = 2.5 cm]{plots/shower_1.png}
		\end{figure}

		\column{0.5\textwidth}
		
		\footnotesize Ensemble of all possible branchings from parton $i$ at scale $t$

	\end{columns}

	\begin{figure}
		\includegraphics[width = 6 cm]{plots/shower_2.png}
	\end{figure}
		
	\footnotesize Forward evolution equation $\rightarrow$ recursive structure
	\begin{figure}
		\includegraphics[width = 10 cm]{plots/shower_2_eq.png}
	\end{figure}

\end{frame}

% FSR III - Probabilistic interpretation
\begin{frame}

	\frametitle{Final state radiation MC}
	\framesubtitle{Probabilistic interpretation}

	\begin{columns} 
	
		\column{0.3\textwidth}
		
		\begin{figure}
			\includegraphics[width = 2.5 cm]{plots/probability1.png}
		\end{figure}
		
		\column{0.7\textwidth}
		
		\footnotesize Probability of branching in the infinitesimal volume $dt'\; dz\; d\phi$

	\end{columns}

	\begin{columns} 
	
		\column{0.5\textwidth}
		
		\begin{figure}
			\includegraphics[width = 4 cm]{plots/probability2.png}
		\end{figure}
		
		\column{0.5\textwidth}
		
		\footnotesize Probability of branching in the interval dt'

	\end{columns}

	\begin{columns} 
	
		\column{0.6\textwidth}
		
		\begin{figure}
			\includegraphics[width = 5.5 cm]{plots/probability3.png}
		\end{figure}
		
		\column{0.4\textwidth}
		
		\footnotesize Probability of no branching in the interval $dt'$
	
	\end{columns}

	\begin{columns} 
	
	\column{0.6\textwidth}
	
	\begin{figure}
		\includegraphics[width = 6 cm]{plots/probability4.png}
	\end{figure}
	
	\column{0.4\textwidth}
	
	\footnotesize Sudakov form factor
	
	\end{columns}
	
	\begin{columns} 
		
		\column{0.4\textwidth}
		
		\begin{figure}
			\includegraphics[width = 4 cm]{plots/probability5.png}
		\end{figure}
		
		\column{0.6\textwidth}
		
		\footnotesize Probability of, starting at $t$, first branching in the phase space element $dt'\; dz\; d\phi$
		
	\end{columns}

\end{frame}

% FSR IV - Shower algorithm
\begin{frame}

	\frametitle{Final state radiation MC}
	\framesubtitle{Shower algorithm}
	
	\footnotesize Generate hard process with probability proportional to its parton level cross section \\
	\footnotesize For each final state colored parton:
	\begin{enumerate} 
		\item \footnotesize Set scale $t = Q$, hard scale of the process
		\item \footnotesize Generate random number $0 < r < 1$
		\item \footnotesize Solve $r = \Delta_{i}(t, t')$ for $t'$
		\item \footnotesize i) if $t' < t_{0}$, no further branching and stop shower
		\item \footnotesize ii) if $t' \geq t_{0}$, generate $j, l$ with energies $$E_{j} = zE_{i} \quad \textrm{and}\quad E_{l} = (1 - z)E_{i}, $$ with $z$ following a distribution given by  $P_{i, jl}(z)$ and with azimuth $\phi$ uniformly distributed in the interval $[0, 2\pi]$.
		\item \footnotesize For each branched partons set $t = t'$ and start from (2)
	\end{enumerate}

\end{frame}

% ISR I
\begin{frame}

\frametitle{Initial state radiation MC}
\framesubtitle{General structure}
	
	\begin{columns}
	
		\column{0.6\textwidth}

		\begin{itemize}
			\item \footnotesize QCD coupling much larger than QED
			\item \footnotesize Coupling grows for small momentum transfer
		\end{itemize}

		\column{0.4\textwidth}
		
		 \footnotesize \color{red}We can never neglect QCD ISR
		 
	\end{columns}

	\vspace{0.5cm}

	\center \footnotesize Radiation from some initial state leading to some hard collision
	\begin{columns}
		
		\column{0.3\textwidth}
		
		\begin{figure}
			\includegraphics[width = 2.5 cm]{plots/ISR_shower0.png}
		\end{figure}
		
		\column{0.6\textwidth}
		
		\begin{itemize}
			\item \footnotesize Initial parton is on shell
			\item \footnotesize ISR showers are spacelike
			\item \footnotesize Parton with scaled momentum $zp$ acquires negative virtuality
		\end{itemize}
	
	\end{columns}

	\vspace{0.25cm}
	
	\center \footnotesize Factorization formula
	\begin{figure}
		\includegraphics[width = 5.5 cm]{plots/ISR_shower0_eq.png}
	\end{figure}

\end{frame}

% ISR II
\begin{frame}
	
	\frametitle{Initial state radiation MC}
	\framesubtitle{Formal representation}
		
	\begin{columns}

		\column{0.5\textwidth}		
	
		\begin{figure}
			\includegraphics[width = 4.5 cm]{plots/ISR_shower_1b.png}
		\end{figure}
	
		\column{0.5\textwidth}

			\footnotesize $\delta x \; S_{i}(m, x, t, E)$ represents ensemble of all possible states containing a spacelike parton $m$ with energy between $xE$ and $(x + \delta x)E$ and scale $t$
			
	\end{columns}

	\begin{itemize}
		\item \footnotesize Procedure for evolution equation as in FSR (with spacelike showers)
		\item \footnotesize Difference from FSR in Sudakov factors and Splitting functions arise at NLO
	\end{itemize}
	
	\footnotesize Forward evolution equation
	\begin{figure}
		\includegraphics[width = 8.5 cm]{plots/ISR_shower_2.png}
	\end{figure}

\end{frame}

% ISR III
\begin{frame}

	\frametitle{Initial state radiation MC}
	\framesubtitle{Backwards evolution equation}

	\footnotesize Great amount of computation time to generate configurations leading to the hard scattering that we want \\
	\footnotesize Moderns MC programs $\rightarrow$ recursive procedure starting at the large scale \\
	
	\begin{columns}
		
		\column{0.5\textwidth}		
		
		\begin{figure}
			\includegraphics[width = 6 cm]{plots/ISR_backwards.png}
		\end{figure}
		
		\column{0.5\textwidth}
		
		\begin{figure}
			\includegraphics[width = 3.5 cm]{plots/ISR_backwards_eq1.png}
		\end{figure}
		
	\end{columns}
	
	\vspace{0.5cm}
	
	\begin{itemize}
		\item \footnotesize The {\color{red}blob I} at the splitting vertex is given by the inclusive splitting kernel $P_{jm}$ \\
		\item \footnotesize Backwards evolution equation (scale dependent parton density)
	\end{itemize}

	\begin{figure}
		\includegraphics[width = 10 cm]{plots/ISR_backwards_eq2.png}
	\end{figure}

\end{frame}

% ISR MC III
\begin{frame}

	\frametitle{Initial state radiation MC}
	\framesubtitle{Shower algorithm}
	
	\footnotesize Generate hard process with probability proportional to its parton level cross section \\
	\footnotesize For each initial state colored parton:
	\begin{enumerate}
		\item Set scale $t = Q$, hard scale of the process
		\item Generate random number $0 < r < 1$
		\item Solve $$r = \frac{f^{(h)}_{i}(t', x) \Delta_{i}(t, t')}{f^{(h)}_{i}(t, x)} \quad \textrm{for }t'$$
		\item \footnotesize i) if $t' < t_{0}$, no further branching and stop shower
		\item \footnotesize ii) if $t' \geq t_{0}$, $j, l$ generate $j$ and $z$ following a distribution given by  $P_{ij}(z)$, and  azimuth $\phi$ uniformly distributed in the interval $[0, 2\pi]$ \\
		Call $l$ radiated parton, assign energies $$E_{j} = zE_{i} \quad \textrm{and}\quad E_{l} = (1 - z)E_{i}, $$
		\item \footnotesize For parton $j$, set $t = t'$ and start from (2)
		\item \footnotesize For parton $l$, set $t = t'$ and proceed with timelike shower as described in FSR
	\end{enumerate}
\end{frame}

% Hadronization I
\begin{frame}
	
	\frametitle{Hadronization}
	\framesubtitle{Some basics}

	\footnotesize A parton becomes a measurable hadron through the emission of a partonic shower \\
	\footnotesize Large number of color approximation 
	\begin{itemize}
		\item \footnotesize Color indices ranging from 1 to $N_{c}$ and keep only dominant contribution
		\item \footnotesize Each parton identified by a unique label
	\end{itemize}

	\begin{figure}
		\includegraphics[width = 8.5 cm]{plots/hadronization.png}
	\end{figure}

\end{frame}

% Hadronization I
\begin{frame}

\frametitle{Hadronization}
\framesubtitle{Some basics}
	
	\footnotesize Hadronization models
	\begin{itemize}
		\item Cluster models
		\begin{itemize}
			\item \footnotesize Gluons are forced to split in quark-antiquark pair
			\item \footnotesize Then decay each color connected quark-antiquark pair independently
			\item \footnotesize Preconfinement, assuming subsystems of color singlet partons
			with universal invariant mass distribution (power suppressed at high masses)	
		\end{itemize}
		\item Lund string model
		\begin{itemize}
			\item \footnotesize Color connected partons collected in a system
			consisting of a quark, several intermediate gluons, and an antiquark
			\item \footnotesize Intermediate gluons are transverse kicks of a continuum medium
		\end{itemize}
	\end{itemize}

	\footnotesize Fragmentation models are one of the most complex aspects of Shower Monte Carlo \\
	\footnotesize These models have unavoidably a large number of parameters

\end{frame}
% Summary
\begin{frame}

	\frametitle{Summary}

\begin{itemize}
	\item \footnotesize LHC processes factorization in perturbative and non-perturbative components
	\item \footnotesize Perturbative QCD applied only at high energies
	\item \footnotesize Monte Carlo shower algorithms describe the enhanced emissions produced by initial and final state partons
	\item \footnotesize Hadronization models describing the non-perturbative physics
\end{itemize}

\end{frame}

\end{document}