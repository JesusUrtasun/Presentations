\documentclass[aspectratio=43]{beamer}
\usepackage[latin1]{inputenc}
\usepackage{amsmath}
\usepackage{amsfonts}
\usepackage{amssymb}
\usepackage{makeidx}
\usepackage{graphicx}
\usepackage{array}

% Customization
\mode<presentation>{
\usetheme{CambridgeUS}
\usecolortheme{dolphin}
\setbeamertemplate{navigation symbols}{}
}

% Define colors
\definecolor{darkgreen}{rgb}{0.0, 0.5, 0.13}
\definecolor{darkblue}{rgb}{0.0, 0.0, 0.55}
\definecolor{darkred}{rgb}{0.55, 0.0, 0.0}

%************************************************************************************************************

% Title and author
\title[QCD and Monte Carlo]{QCD and Monte Carlo}
\author{\textbf {Jes\'us Urtasun Elizari}}
\date{Milan, February 2021}

\begin{document}

% Front slide
\begin{frame}
	
	%\maketitle
	\vspace{1.0 cm}
	
	\center{\color{blue}QCD and Monte Carlo event generators}
	
	\vspace{0.25 cm}
	\center{Monte Carlo course seminar - Milan, February 2021}

	\begin{figure}
		\minipage{1\textwidth}
		\includegraphics[width = 3.0 cm]{plots/logo_unimi.png}
		\hfill
		\includegraphics[width = 3.0 cm]{plots/logo_infn.png}
		\hfill
		\includegraphics[width = 3.0 cm]{plots/logo_erc.png}
		\endminipage
	\end{figure}

	\vspace{1.0 cm}

\end{frame}

% Introduction
\begin{frame}

	\frametitle{Outline}
	
	\begin{enumerate}
		\item {\color{blue}Hadron collisions and strong interactions}
		\begin{itemize}
			\item Hadron collisions and strong interactions
			\item Renormalization group
			\item IR divergences
		\end{itemize}
		\item {\color{blue}MC and Parton Showers}
		\begin{itemize}
			\item Factorization theorem
			\item Final state radiation
			\item Initial state radiation
		\end{itemize}
		\item {\color{blue}Hadronization: some basics}
	\end{enumerate}
	
\end{frame}

% Strong interactions I
\begin{frame}

	\frametitle{Strong interactions}
	\framesubtitle{QCD from $e^{+}e^{-}$ annihilation}

	Quantum Chromodynamics (QCD) $\rightarrow$ theory describing the interaction between quarks and gluons (strong interactions)
	\begin{figure}
		\includegraphics[width = 5 cm]{plots/ee_hadrons.png}
	\end{figure}
 
	QCD arises already from $e^{+}e^{-}$ annihilation $\rightarrow R_{0}$ ratio

	\begin{columns}
	
		\column{0.5\textwidth}
	
		\begin{figure}
			\includegraphics[width = 6 cm]{plots/eq_R0.png}
		\end{figure}
	
		\column{0.5\textwidth}
	
		\begin{enumerate}
			\item \footnotesize Color factor (3 color for each quark)
			\item \footnotesize Sum over charges of different flavors
			\item \footnotesize Threshold and higher order corrections
		\end{enumerate}	

	\end{columns}

\end{frame}

% Strong interactions II
\begin{frame}

	\frametitle{Strong interactions}
	\framesubtitle{QCD from $e^{+}e^{-}$ annihilation}
	
	Higher order corrections to $R_{0}$ 
	\begin{figure}
		\includegraphics[width = 5 cm]{plots/ee_hadrons.png}
	\end{figure}
	
	QCD arises already from $e^{+}e^{-}$ annihilation $\rightarrow R_{0}$ ratio
	
	\begin{columns}
		
		\column{0.5\textwidth}
		
		\begin{figure}
			\includegraphics[width = 6 cm]{plots/eq_R0_3.png}
		\end{figure}
		
		\column{0.5\textwidth}
		
		\begin{enumerate}
			\item \footnotesize Color factor (3 color for each quark)
			\item \footnotesize Sum over charges of different flavors
			\item \footnotesize Threshold and higher order corrections
		\end{enumerate}	
		
	\end{columns}

\end{frame}

% Renormalization group I
\begin{frame}

	\frametitle{Strong interactions}
	\framesubtitle{Renormalization group}
	
	\begin{itemize}
		\item Running coupling given by Renormalization Group Equation (RGE)
		\begin{equation}
			{\color{blue}\mu\frac{d\alpha_{s}(\mu)}{d\mu} = \beta(\alpha_{s}(\mu)) = -\sum_{n = 0}^{\infty} \beta_{n} \Big( \frac{\alpha_{s}}{\pi} \Big)^{n + 1}} \nonumber
		\end{equation}
		\item Coupling {\color{blue}$\alpha_{s}$} evolves with scale {\color{blue}$\mu$} as given by RGE $\rightarrow$ LO behavior driven by $\beta_{0}$
		\item $\beta_{0}^{\textrm{QCD}} > 0 \implies$ weakly coupled at large energies, asymptotic freedom
		\item $\beta_{0}^{\textrm{QED}} < 0 \implies$ strongly coupled at large energies, UV divergent!	
	\end{itemize}

\end{frame}

% Renormalization group II
\begin{frame}

	\frametitle{Strong interactions}
	\framesubtitle{Renormalization group}

	\begin{columns}	
	
		\column{0.5\textwidth}
		
		\begin{figure}
			\includegraphics[width = 5 cm]{plots/qcd_coupling.png}
		\end{figure}

		\column{0.5\textwidth}
	
		\begin{itemize}
			\item \footnotesize Running coupling given by Renormalization Group Equation (RGE)
			\begin{equation}
				\alpha_{s}(\mu) = \frac{1}{\beta_{0} \log\big( \frac{\mu^{2}}{\Lambda_{s}^{2}}\big)} \nonumber
			\end{equation}
			\item \footnotesize $\beta_{0}$ LO of the $\beta$ function, is $ > 0$
			\item \footnotesize $\Lambda_{s}$, parameter that defines value of the coupling at large scales
		\end{itemize}

	\end{columns}
	
	\vspace{1cm}
	\center QCD is weakly coupled for $\mu >> \Lambda_{s} \longrightarrow$ asymptotically free
	\center \color{red} Perturbative Quantum Chromodynamics (pQCD)

\end{frame}

% Factorization theoren
\begin{frame}

	\frametitle{Factorization theorem}
	\framesubtitle{QCD factorization}
	
	\center \footnotesize LHC processes $H_{1} + H_{2} \rightarrow \textrm{F}$	
	\begin{figure}
		\includegraphics[width = 7 cm]{plots/factorization_1.png}
	\end{figure}
	
	\footnotesize {Separate process {\color{blue}PDFs} and {\color{red} partonic (hard) interaction}	
	\begin{equation}
		\sigma^{\textrm{F}}(p_{1}, p_{2}) = \sum_{\alpha, \beta}
		\int_{0}^{1} dx_{1} dx_{2} \; {\color{blue} f_{\alpha}(x_{1}, \mu_{F}^{2}) \ast f_{\beta}(x_{2}, \mu_{F}^{2})}
		\; \ast \;  
		{\color{red}\hat{\sigma}^{\textrm{F}}_{\alpha \beta}(x_{1}p_{1}, x_{2}p_{2}, \alpha_{s}(\mu_{R}^{2}), \mu_{F}^{2})} \nonumber
	\end{equation}}
		
\end{frame}

% Parton showers I
\begin{frame}

	\frametitle{Parton showers}
	\framesubtitle{MC Parton showers}
	
	\footnotesize Partons in the initial and final state emit radiation. Initial state Radiation (ISR) and Final State Radiation (FSR) model by Monte Carlo (MC) shower algorithms.
	
	\vspace{0.1 cm}
	\center \color{red} Shower Monte Carlo programs (HERWIG, PYTHIA)
	\vspace{0.15 cm}
	
	\begin{itemize} 
		\item Libraries for computing SM and BSM cross sections
		\item Shower algorithms produce the parton shower from final state or initial state partons (accurate only at LO?...)
		\item Hadronization models, underlying event, decays of unstable hadrons, etc
	\end{itemize}

\end{frame}


% Parton showers II
\begin{frame}

	\frametitle{Parton showers}
	\framesubtitle{Collinear limit}
	
	\begin{itemize} 
		\item \footnotesize An emitted parton is collinear to an incoming or outgoing parton ($\theta$ small)
		\item \footnotesize $\sigma$ dominated by collinear emission $q \rightarrow qg, g \rightarrow gg, g \rightarrow q\bar{q}$ \\
		(measurement not sensitive to such small scales)
	\end{itemize}
	
	\begin{figure}
		\includegraphics[width = 6.5 cm]{plots/collinear_factorization.png}
	\end{figure}
	
	\footnotesize Collinear factorization $\longrightarrow$ Factor out tree level amplitude and splitting
	\begin{figure}
		\includegraphics[width = 7 cm]{plots/eq_factorization_theorem.png}
	\end{figure}

	\begin{figure}
		\includegraphics[width = 5 cm]{plots/eq_factorization_ps.png}
	\end{figure}

\end{frame}

% Parton showers III
\begin{frame}

	\frametitle{Parton showers}
	\framesubtitle{Kinematics of splitting}
	
	\begin{figure}
		\includegraphics[width = 5 cm]{plots/shower_kinematics.png}
	\end{figure}
	
	\footnotesize Kinematics of splitting given by $(t, z, \phi)$
	\begin{itemize}
		\item $\phi$ represents azimuth of the $k, l$ plane
		\item $z$ is the fraction of energy of radiated parton $$z = \frac{k^{0}}{k^{0} + l^{0}}$$
		\item $t$ has dimensions of energy \\
			- virtuality $$t = (k + l)^{2} = k^{0}l^{0}4\sin^{2}\bigg( \frac{\theta}{2} \bigg) \approx k^{0}l^{0}\theta^{2} \approx z(1 - z)E^{2}\theta^{2}$$ \\
			- transverse momentum $t = k_{\perp}^{2} = l_{\perp}^{2} = z^{2}(1 - z)^{2}E^{2}\theta^{2}$ \\
			- hardness $E^{2}\theta^{2}$ \\
	\end{itemize}
	
	\vspace{0.5cm}
	
	Factorization holds for small angles. Applied recursively
	
\end{frame}

% Parton showers IV
\begin{frame}

	\frametitle{Parton showers}
	\framesubtitle{AP splitting functions}
	
	\begin{columns}
	
		\column{0.4\textwidth}
			 
		\footnotesize Altarelli-Parisi splitting functions	
		\begin{figure}
			\includegraphics[width = 5 cm]{plots/AP_splitting.png}
		\end{figure}
		
		\column{0.6\textwidth}
		
		\begin{figure}
			\includegraphics[width = 6.5 cm]{plots/AP_iteration_plot.png}
		\end{figure}

	\end{columns}

	\vspace{0.4cm}
	
	\footnotesize We can proceed in an iterative way	
	\begin{figure}
		\includegraphics[width = 9 cm]{plots/AP_iteration_eq.png}
	\end{figure}

	\footnotesize Exclusive final state: limit to the most singular terms, in ordered sequence of angles
	\color{red} Collinear approximation $\longrightarrow$ Leading log approximation
	
\end{frame}

% Parton showers V
\begin{frame}
	
	\frametitle{Parton showers}
	\framesubtitle{Exclusive final state}
	
	\footnotesize Exclusive final state: sum the perturbative expansions to all orders in $alpha_s$ \\
	\footnotesize Limit to the most singular terms in ordered sequence of angles $alpha_s$
	
	\begin{figure}
		\includegraphics[width = 9 cm]{plots/exclusive_ll.png}
	\end{figure}
	
	\footnotesize Exclusive final state: limit to the most singular terms, in ordered sequence of angles
	\color{red} Collinear approximation $\longrightarrow$ Leading log approximation

\end{frame}

% FSR I
\begin{frame}

	\frametitle{Final state radiation MC}
	\framesubtitle{General structure}
	
	\footnotesize Approximated description of a hadronic final state \\
	\footnotesize Model a given hard scattering with arbitrary number of enhanced radiations

	\begin{itemize} 
		\item Choose hard interaction with specified Born kinematics.
		\item Consider all possible splittings for each coloured parton.
		\item Assign the variables t, z, $\phi$ at each splitting vertex, t ordered in decreasing way.
		\item At each splitting vertex assign the weight 
		\begin{figure}
			\includegraphics[width = 2.5 cm]{plots/AP_weight.png}
		\end{figure}
		\item Each line has a weight known as Sudakov factor
		\begin{figure}
			\includegraphics[width = 6 cm]{plots/sudakov.png}
		\end{figure}
	\end{itemize}

\end{frame}

% FSR II - Formal representation of a shower
\begin{frame}

	\frametitle{Final state radiation MC}
	\framesubtitle{Formal representation of a shower}
	
	\footnotesize Approximated description of a hadronic final state \\
	\footnotesize Model a given hard scattering with arbitrary number of enhanced radiations
	
	\begin{columns} 
		
		\column{0.5\textwidth}
		
		\begin{figure}
			\includegraphics[width = 2.5 cm]{plots/shower_1.png}
		\end{figure}

		\column{0.5\textwidth}
		
		\footnotesize Ensemble of all possible branchings at scale t (...)

	\end{columns}

	\begin{figure}
		\includegraphics[width = 6 cm]{plots/shower_2.png}
	\end{figure}
		
	\footnotesize Forward evolution equation
	\begin{figure}
		\includegraphics[width = 10 cm]{plots/shower_2_eq.png}
	\end{figure}

\end{frame}

% FSR III - Probabilistic interpretation
\begin{frame}

	\frametitle{Final state radiation MC}
	\framesubtitle{Probabilistic interpretation}

	\begin{columns} 
	
		\column{0.3\textwidth}
		
		\begin{figure}
			\includegraphics[width = 2.5 cm]{plots/probability1.png}
		\end{figure}
		
		\column{0.7\textwidth}
		
		\footnotesize Probability of branching in the infinitesimal volume $dt'\; dz\; d\phi$

	\end{columns}

	\begin{columns} 
	
		\column{0.5\textwidth}
		
		\begin{figure}
			\includegraphics[width = 4 cm]{plots/probability2.png}
		\end{figure}
		
		\column{0.5\textwidth}
		
		\footnotesize Probability of branching in the interval dt'

	\end{columns}

	\begin{columns} 
	
		\column{0.6\textwidth}
		
		\begin{figure}
			\includegraphics[width = 5.5 cm]{plots/probability3.png}
		\end{figure}
		
		\column{0.4\textwidth}
		
		\footnotesize Probability of first branching in the infinitesimal volume $dt'$
	
	\end{columns}

	\begin{columns} 
	
	\column{0.6\textwidth}
	
	\begin{figure}
		\includegraphics[width = 6 cm]{plots/probability4.png}
	\end{figure}
	
	\column{0.4\textwidth}
	
	\footnotesize Sudakov form factor from unitarity
	
	\end{columns}
	
	\begin{columns} 
		
		\column{0.4\textwidth}
		
		\begin{figure}
			\includegraphics[width = 4 cm]{plots/probability5.png}
		\end{figure}
		
		\column{0.6\textwidth}
		
		\footnotesize Probability of no-branching in the infinitesimal volume $dt'\; dz\; d\phi$
		
	\end{columns}

\end{frame}

% FSR IV - Shower algorithm
\begin{frame}

	\frametitle{Final state radiation MC}
	\framesubtitle{Shower algorithm}
	
	\footnotesize Generate hard process with probability proportional to its parton level cross section. \\
	\footnotesize For each final state colored parton:
	\begin{enumerate} 
		\item \footnotesize Set scale $t = Q$, hard scale of the process.
		\item \footnotesize Generate random number $0 < r < 1$.
		\item \footnotesize Solve $r = \Delta_{i}(t, t')$ for $t'.$
		\item \footnotesize i) if $t' < t_{0}$, no further branching and stop shower.
		\item \footnotesize ii) if $t' \geq t_{0}$, generate $j, l$ with energies $$E_{j} = zE_{i} \quad \textrm{and}\quad E_{l} = (1 - z)E_{i}, $$ following the $P_{i, jl}(z)$ distribution and with azimuth $\phi$ uniform in the interval $[0, 2\pi]$. \\
		the angle of between their momenta is fixed by $t'$.
		\item \footnotesize For each branched partons set $t = t'$ and start from (2).
	\end{enumerate}

\end{frame}

% ISR I
\begin{frame}

	\frametitle{Initial state radiation MC}
	\framesubtitle{General structure}
	
	\center \footnotesize ISR showers are spacelike
	\begin{columns}
		
		\column{0.3\textwidth}
		
		\begin{figure}
			\includegraphics[width = 2 cm]{plots/ISR_shower0.png}
		\end{figure}
	
		\column{0.6\textwidth}
		
		\begin{figure}
			\includegraphics[width = 5 cm]{plots/ISR_spacelike.png}
		\end{figure}
		
	\end{columns}

	\vspace{0.5cm}
	\footnotesize Consider the splitting between a particle a that splits into b and c
	
	\begin{figure}
		\includegraphics[width = 9 cm]{plots/ISR_splitting.png}
	\end{figure}

	\begin{figure}
		\includegraphics[width = 4.5 cm]{plots/ISR_FSR.png}
	\end{figure}

\end{frame}

% ISR II
\begin{frame}
	
	\frametitle{Initial state radiation MC}
	\framesubtitle{Formal representation}
	
	\begin{figure}
		\includegraphics[width = 4 cm]{plots/shower_ISR_1.png}
	\end{figure}
	
	\begin{itemize} 
		\item \footnotesize Lines between $t_{1}$ and $t_{2}$ (consecutive radiations) are spacelike {\color{blue}(*)}
		\item \footnotesize Difference in Sudakov factors and Splitting functions start at NLO
	\end{itemize}
	
	\footnotesize Forward evolution equation. Great amount of computation time to generate configurations leading to the scattering that we want
	\begin{figure}
		\includegraphics[width = 8.5 cm]{plots/shower_ISR_2.png}
	\end{figure}

\end{frame}

% ISR MC III
\begin{frame}

	\frametitle{Initial state radiation MC}
	\framesubtitle{Shower algorithm}
	
	\footnotesize Generate hard process with probability proportional to its parton level cross section.
	\footnotesize For each final state colored parton:
	\begin{enumerate} 
		\item Set scale $t = Q$, hard scale of the process.
		\item Generate random number $0 < r < 1$
		\item Solve $$r = \frac{f^{(i)}_{m} \Delta_{m}(t, t')}{f^{(i)}_{m}(t, x)} \quad \textrm{for }t'.$$
		\item i) if $t' < t_{0}$, no further branching and stop shower.
		\item \footnotesize ii) if $t' \geq t_{0}$, generate $j, l$ with energies $$E_{j} = zE_{i} \quad \textrm{and}\quad E_{l} = (1 - z)E_{i}, $$ following the $P_{i, jl}(z)$ distribution and with azimuth $\phi$ uniform in the interval $[0, 2\pi]$. \\
		the angle of between their momenta is fixed by $t'$.
		\item For parton $j$ set $t = t'$ and start from (2). For parton $l$ generate a timelike parton shower according to the algorithm shown previously.
	\end{enumerate}

\end{frame}

% Hadronization I
\begin{frame}
	
	\frametitle{Hadronization}
	\framesubtitle{Basics}

	\footnotesize A parton becoming a measurable hadron through the emission of a partonic shower \\
	\footnotesize Large number of color approximation $\rightarrow$ each parton identified by a unique label \\

	\begin{figure}
		\includegraphics[width = 8.5 cm]{plots/hadronization.png}
	\end{figure}

	\footnotesize Hadronization models
	\begin{itemize}
		\item Lund string model
		\begin{itemize}
			\item \footnotesize non perturbative production of quarks and antiquarks
			\item \footnotesize intermediate gluons are transverse kicks of a continuum medium
		\end{itemize}
		\item Cluster models
		\begin{itemize}
			\item \footnotesize preconfinement, assuming subsystems of color singlet partons
			with universal invariant mass distribution (power suppressed at high masses)
			\item gluons are forced to split in quark-antiquark pair
		\end{itemize}
	\end{itemize}

\end{frame}

% Summary
\begin{frame}

	\frametitle{Summary}

\begin{itemize}
	\item \footnotesize LHC processes require factorization in perturbative and non perturbative part
	\item \footnotesize pQCD applied at high energies
	\item \footnotesize Monte Carlo shower programs describe non perturbative physics in hadron physics
	\item \footnotesize Agreement and precision Monte Carlo shower programs
\end{itemize}

\end{frame}

\end{document}