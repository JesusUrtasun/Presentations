\documentclass[12pt]{article}
%%%%%%%%%%%%%%%%%%%%%%%%%%%%%%%%%%%%%%%%%%%
\usepackage{graphicx}
\usepackage{epsfig}
\usepackage{subfig}
\usepackage{amsmath}
\usepackage{amssymb}
\usepackage{geometry}
\usepackage{indentfirst}
\geometry{
	a4paper,
	total={170mm,257mm},
	left=20mm,
	top=20mm,
}
\usepackage{xcolor}
% For correct labelling
%(...)
\usepackage{hyperref}
%%%%%%%%%%%%%%%%%%%%%%%%%%%%%%%%%%%%%%%%%%%

\begin{document}
	
\title{Research Activity Report}
\author{PhD student: Jes\'us Urtasun Elizari \\
		Supervisor: Prof. Giancarlo Ferrera \\
		Field of Research: Theoretical Physics}
\date{Milan, September 16th, 2021}
\maketitle

\section{Introduction}

The argument of my PhD lies on Standard Model (SM) physics, and in the accurate description of matter and its interactions at the fundamental scales. It is based on both Computation Sciences and Quantum Chromodynamics (QCD), oriented to LHC physics. When addressing LHC and Standard Model physics, for which extremely precise predictions exist up to the TeV scale, reducing the theoretical uncertainties has become a main task for theoreticians, specially for the precision era of the LHC. While the first part of my PhD was devoted to the precise determination of the Parton Distribution Functions, the second part has focused on developing a new numerical code, HTurbo, which aims to produce fast and numerically precise predictions for the Higgs boson at the LHC. Both projects are summarized in this report.\\

\section{Machine Learning for the precise determination of the Parton Distribution Functions}
The first year of my PhD, within the NNPDF Collaboration, was devoted to Machine Learning applications to the precise numerical determination of the Parton Distribution Functions (PDFs), which describe the probability distributions of the quarks and gluons inside the proton. PDFs are crucial for the computation of the SM observables required for the LHC physics, and hence they cannot be precisely calculated from first principles. Therefore, the use of such artificial intelligence algorithms provides a powerful set of techniques towards the production of precise and reliable predictions. In the proceedings linked at the end of this report, we described and quantified the most promising data-model configurations to increase PDF fitting performance, such as the implementation of C++ operators within the TensorFlow library and adapting the current code frameworks to hardware accelerators such as graphics processing units.\\

I also worked on a subgroup of Machine Learning algorithms, the so-called Gaussian Processes and Hyper-optimization, to interpolate partonic distributions and reduce theory uncertainties by training on their asymptotic expressions. Preliminary results are shown in the following contribution at the Italian Physical Society, where the Hyper-optimized Gaussian Process regressor is shown to build excellent numerical agreement up to next-to-next-to- leading order (NNLO) accuracy on inclusive observables.

\section{Accurate theoretical predictions for Higgs boson physics at the LHC}
From the second year of my PhD I started to work on perturbative QCD phenomenology at hadron colliders, in particular on Sudakov resummation, that is, resummation of double-logarithmic perturbative contributions produced by soft-gluon radiation. Our main goal goal is to provide fast and numerically precise predictions through the factorization of the cross section into production and decay variables, to be used towards the next runs of the LHC, where the accuracy of predictions and the minimization of theory uncertainties has become the main task for theoreticians.\\

Cross-sections in particle physics processes are typically computed using perturbation theory, which provides a very powerful set of tools to predict observable quantities from a Quantum Field Theory. It is based on the assumption that a given observable can be defined by a power series in the coupling constant of the theory: then, if such coupling is small enough to be considered a \textit{perturbative} parameter, the computation of the first few orders of the power series is sufficiently accurate to describe the observable. However, this assumption needs some care, since perturbative series can indeed de divergent. It is possible to interpret the perturbative expansion in the sense of an asymptotic series, where the inclusion of higher orders improves the estimate of the physical quantity under study, up to some finite order, thereby saving perturbation theory from a dramatic failure. However, there are situations in which the growth of the series already starts at the level of the first terms; in these cases, a truncation of the series is of no meaning and only a \textit{resummed} result is reliable. This is a situation that quite often appears in QCD, and hence one of the main problems addressed in this thesis. A cross-section generically depends on many energy scales, and the dependence is typically in the form of logarithmic ratios of energies. In some kinematical regimes, when two of such scales become very different each other these logs become large and the coefficients of the perturbative series are significantly enhanced, making the standard perturbative approach unreliable. Then, the entire series of these enhanced terms has to be resummed in order to have an accurate prediction for the observable.\\

Therefore, the production of fast and accurate - resummed - predictions has become one of the main tasks among theoreticians, heading towards the high luminosity era of the LHC, and resummation in QCD is needed for the prediction of hadronic observables, and in particular of differential distributions. The HTurbo program provides fast and numerically precise predictions for Higgs boson production, through a new implementation of the HqT, HRes and NNLO numerical codes. The cross-section predictions include the calculation of the QCD transverse-momentum resummation up to next-to-next-to-leading logarithmic accuracy combined with the fixed-order results at next-to-next-to-leading order. They also include the full kinematical dependence of the decaying photon pair. The enhancement in performance over previous programs is achieved by code optimization, by factorizing the cross section into production and decay variables, and with the usage of numerical integration based on interpolating functions. With paper in preparation, the resulting cross-section predictions are in agreement with the results of the original programs at the per mille level. The great reduction of computing time for performing cross-sections calculation opens new possibilities for Higgs processes at the LHC, and also opens the possibility of searches for physics beyond the SM.

\newpage

\section{Formations activities}

\subsection{UNIMI Courses}

\begin{enumerate}{\leftmargin 15pt \itemsep 0pt \topsep 3pt}
	\item{\bf Computational, simulation and machine learning methods in high energy physics and beyonf: Automated Computational Tools (3 CFU - 15 hrs, Prof. Fabio Maltoni)}
	\item{\bf Computational, simulation and machine learning methods in high energy physics and beyonf: Monte Carlo Methods (3 CFU - 15 hrs, Prof. Paolo Nason)}
	\item{\bf Computational, simulation and machine learning methods in high energy physics and beyonf: Machine Learning Tools (3 CFU - 15 hrs, Dr. Stefano Carrazza)}
	\item{\bf Quantum Coherent Phenomena (6 CFU - 15 hrs, Prof. Fabrizio Castelli, Dr. Marco Geoni, Dr. Claudia Benedetti)}
\end{enumerate}

\subsection{International Schools}

\begin{enumerate}{\leftmargin 15pt \itemsep 0pt \topsep 3pt}
	\item{\bf GGI School on Theory of the Fundamental Interactions. Galileo Galilei Institute, Florence, Italy, January 2019. \\ \href{https://www.ggi.infn.it/ggilectures/ggilectures2019/}{https://www.ggi.infn.it/ggilectures/ggilectures2019/}}
	\item{\bf CTEQ summer schools on QCD and EW Phenomenology. University of Pittsburgh, Pennsylvania, USA, July 2019. \\ \href{https://indico.fnal.gov/event/43762/overview}{https://indico.fnal.gov/event/43762/overview}}
	\item{\bf CERN-Fermilab summer school on Hadron Collider Physics (HCPSS). Fermilab (via zoom), August 2020. \\ \href{https://inspirehep.net/conferences/1713348}{https://inspirehep.net/conferences/1713348}}
\end{enumerate}

\subsection{Teaching activities}

\begin{enumerate}{\leftmargin 15pt \itemsep 0pt \topsep 3pt}
	\item{\bf Teaching assistant for \textit{Informatica - Computer Sciences} \hspace{1cm} Oct-Feb 2019}
	\item{\bf Teaching assistant for \textit{Meccanica Quantistica} \hspace{3.5cm} Mar-Jul 2021}
\end{enumerate}

\section{Publications and Conferences}

\begin{enumerate}{\leftmargin 15pt \itemsep 0pt \topsep 3pt}
	\item{\bf Towards hardware acceleration for parton densities estimation - Photon 19 \href{https://arxiv.org/abs/1909.10547}{https://arxiv.org/abs/1909.10547}}
	\item{\bf Machine Learning for the determination of Parton Distribution
		Functions - Institute of High Energy Physics IFAE Barcelona \\ \href{https://indico.ifae.es/event/645/}{https://indico.ifae.es/event/645/}}
	\item{\bf Gaussian Processes for the estimation of theory uncertainties - Italian Physical Society SIF National congress 2020\\ \href{https://agenda.infn.it/event/23656/contributions/120309/}{https://agenda.infn.it/event/23656/contributions/120309/}}
	\item {\bf HTurbo: Fast and numerically precise predictions for Higgs boson physics at the LHC - paper in preparation}
\end{enumerate}

\newpage


\hspace{2cm}


PhD student: Jesús Urtasun\\

\hspace{2cm}

\indent Supervisor: Giancarlo Ferrera
\end{document}